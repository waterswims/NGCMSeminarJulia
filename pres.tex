\documentclass{beamer}
\usepackage{xcolor}
%\usepackage{fontspec}
%\setmainfont{Georgia}
%\usepackage[T1]{fontenc}
\usepackage{winfonts}
\usepackage[backend=bibtex]{biblatex}
\renewcommand{\sfdefault}{georgia}

\bibliography{report}

%%%%%%%%%%%%%%
% Soton colourscheme

%primary palette:           RED    GREEN  BLUE
\definecolor{sotonblu}{rgb}{.00392 .26275 .34902} % soton blue
\definecolor{sotongrn}{rgb}{.00000 .44706 .45882} % soton green
\definecolor{sotoncya}{rgb}{.03922 .58824 .66275} % soton cyan
\definecolor{sotongry}{rgb}{.19608 .23922 .26275} % soton grey
\definecolor{sotonbei}{rgb}{.59216 .61961 .27059} % soton beige
\definecolor{sotonmet}{rgb}{.73333 .73333 .73333} % soton metal

%some secondary colors:
\definecolor{sotonyel}{rgb}{.99999 .70196 .00000} % soton yellow
\definecolor{sotonora}{rgb}{.99608 .24314 .07843} % soton orange
\definecolor{sotonred}{rgb}{.94118 .05882 .17255} % soton red
\definecolor{sotonrus}{rgb}{.67059 .07059 .06275} % soton russet
\definecolor{sotonbrn}{rgb}{.54118 .25490 .16863} % soton brown
\definecolor{sotonpnk}{rgb}{.88627 .41176 .62353} % soton pink
\definecolor{sotonppl}{rgb}{.32549 .12157 .26667} % soton purple


\setbeamertemplate{background canvas}[vertical shading][top=sotonblu,bottom=sotoncya]
\setbeamercolor{background canvas}{bg=}
\setbeamercolor{button border}{bg=sotonblu, fg=sotonblu}
\setbeamercolor{button}{bg=sotonblu, fg=DarkRed}

\setbeamercolor{frametitle}{fg=sotonyel}
\setbeamercolor{alerted text}{fg=sotonyel}
\setbeamercolor{normal text}{fg=white}
\setbeamercolor{titlelike}{fg=sotonyel}
\setbeamercolor{author}{fg=white}
\setbeamercolor{date}{fg=white}
\setbeamercolor{item}{fg=white}

%%%%%%%%%%%%%%%%%%%%

\title{Why Julia?}
\author{Jonathon Waters}
\institute{
	Cohort 1,\\
	EPSRC CDT in Next Generational Computational Modelling,\\
	University of Southampton
}
\date{}

\begin{document}
\frame{\titlepage
\includegraphics[height=1cm]{Images/sponsor-wo}\hfill\includegraphics[height=1cm]{Images/uos_grey_large}}

\setbeamertemplate{headline}{
	\vskip25pt % horizontal line
	\vskip-21pt\hspace{2pt}\hfill\includegraphics[height=7mm]{Images/sponsor-wo}\hspace{3.5mm}\includegraphics[height=7mm]{Images/uos_grey_large}\hspace{3.5mm} % logo on the right
}

\setbeamertemplate{footline}{
	\vskip-8pt % horizontal line
	\hspace{12cm} \insertframenumber/\inserttotalframenumber
	\vskip4pt
}

\begin{frame}
	\frametitle{Outline of Today's Talk}
	\begin{itemize}
		\item What is Julia?
		\begin{itemize}
			\item The Goals of Julia
		\end{itemize}
		\item Key Features of Julia
		\begin{itemize}
			\item Multiple Dispatch
			\item Performance
			\item Built-in Package Manager
			\item Calls to Other Languages
			\item Parallelism
		\end{itemize}
		\item Competitors
		\item ...
	\end{itemize}
\end{frame}

\begin{frame}
	\frametitle{What is Julia?}
	From julialang.org: \newline

	\textit{``Julia is a high-level, high-performance dynamic programming language for technical computing, with syntax that is familiar to users of other technical computing environments.''}

	\vspace{5mm}\textit{``It provides a sophisticated compiler, distributed parallel execution, numerical accuracy, and an extensive mathematical function library.''}
\end{frame}

\begin{frame}
	\frametitle{The Goals of Julia}
	\begin{itemize}
		\item The \textit{speed} of compiled languages (C/C++ and Fortran) \newline
		\item The \textit{dynamism} of high-level languages (Ruby, Python) \newline
		\item The \textit{mathematical notations} of Matlab \newline
		\item The \textit{general usage} of Python \newline
		\item The \textit{statistical ease} of R
	\end{itemize}
\end{frame}

\begin{frame}
	\frametitle{Key Features of Julia}

	\begin{itemize}
		\item Multiple Dispatch \newline
		\item Performance \newline
		\item Built-in Package Manager \newline
		\item Calls to Other Languages \newline
		\item Parallelism \newline
		\item ...
	\end{itemize}
\end{frame}

\begin{frame}
	\frametitle{Thank You}
	\begin{center}
		Thank you all for listening and I welcome any questions. \newline

		Contact: J.M.Waters@soton.ac.uk
	\end{center}
\end{frame}
\end{document}
